
Finally, we will summarize the associations of gut microbes with disease
for all datasets, and compare these across all studies to determine whether certain diseases have 
more similar gut microbial signatures than others.


\subsection{Ignore}
We hypothesize that there is a clinically-relevant exchange of bacteria within the aerodigestive tract that may be altered in certain disease states. 


Another important consideration in this work may be if study-associated effects are larger than biological effects. For example, when we compare 'microbial signatures' across datasets, it's possible that the largest signal driving dataset clustering is sequencer or 16S region sequenced, rather than disease state. There are many approaches we could take to correct for such batch effects:
\begin{enumerate}
	\item Subtracting the principal components corresponding to the technical artifacts.
	\item Build a model that accounts for these technical artifacts by including them as factors in the model.
	\item Non-parametric correction, like sample- or OTU-wise quantile normalization, using controls in each study as the reference distribution.
\end{enumerate}

This work will improve our 
understanding of the clinical relevance of the human microbiome and 
will also provide new approaches and tools for analyzing future 
studies.

We will summarize each dataset's microbial communities
methods commonly used in the 
literature: univariate non-parameteric statistical tests on relative 
abundances, alpha and beta diversity in different types of patients, 
and ratios of Firmicutes to Bacteroides in healthy vs. disease 
patients. It is generally thought that low alpha diversity is a marker 
of dysbiosis (REF), and that while most people have a Firmicutes/
Bacteroides ratio of (XXX), in certain diseases this ratio may be 
different (REF). (MAYBE BACKGROUND?) By analyzing each study in the 
same way from raw data, we can reduce the study-wise batch effects and 
increase our ability to identify general trends in the gut microbiome 
in health and disease. We will identify consistent markers of disease 
by using standard meta-analysis methods, comparing the effect sizes 
and directionality of bacteria across studies, and using Fisher's 
method to determine overall significant of a microbe (REFS). 

"Thus, humans are super organisms: http://science.sciencemag.org/content/312/5778/1355"

The microbiota of 
the aerodigestive tract is poorly studied, and we have little understanding 
of how the microbial communities in different aeordigest sites are related or 
affected by disease. In contrast, the gut microbiome has 
been extensively studied through many case-control studies. 
However, these studies have frequently yielded inconsistent or incomparable 
results. Existing meta-analyses have not extended to more than one or two 
diseases, and thus can not determine whether significant microbes are 
associated with specific diseases or with disease in general. 
Finally, there are no existing tools that can be used to extract general 
biological insights from groups of disease-associated microbes.

Fecal microbiota transplants have demonstrated the incredible
causal ability of the microbiome to affect health in 
human patients. Germ-free mouse models have shown that microbes are
necessary for healthy functioning, and specific animal models have  
allowed for probing mechanistic understanding of host-microbial 
interactions (autism 440 ref, gordon mouse experiment ref). 

Additionally, many of the most successful existing meta-analyses 
combine vastly different types of microbial communities and non-case-
control experimental designs. The positive results from these studies 
are not particularly biologically insightful: it is a much easier task 
to differentiate vastly different communities (like the skin vs. the 
gut) than it is to differentiate subtleties contributing to health and 
disease (like the inflamed gut vs. the healthy gut) \cite{knights-supervised-2010}.

Every meta-analysis performed on 16S data has observed 
strong batch effects between studies and noted the need for large 
sample sizes to extract any meaningful signal \cite{sze-signal-2016},
\cite{walters-ob_meta-2014},\cite{knights-supervised-2010},
\cite{lozupone-meta-2013}. 

This definition depends on correlated abundances and not simply co-occurence, and is 
superior to simple co-occurence because of the significant overlap in the
members of the three aeordigestive communities\cite{bassis-source-2015}, \cite{charslon-topographical-2011}.

To determine how lung, gastric, and throat microbial communities are related,
we will calculate $p_s$ for each of the site combinations. We will 
then see if there are apparent phylogenetic relationships between
the exchanged microbes. We expect to find significantly more exchange
between the throat and stomach, and very little exchange between the throat and lungs. 




\begin{table}
\begin{tabular}{|p{6cm}|p{10cm}|}
	\hline
	\textbf{Category} & \textbf{Approach} \\
	\hline
	Pathogens & Targeted literature search, literature mining, \& 
	databases \\
	\hline
	Body sites & Literature mining, machine learning on HMP data \\
	\hline
	Environmental associations & Literature mining, machine learning 
	on EMP data \\
	\hline
	Growth rate & Inference from 16S sequences in datasets from Aim 2 
	and HMP \\
	\hline
	Obesity-associated & Targeted literature search \& machine 
	learning on Aim 2 datasets \\
	\hline
	Inflammation-associated & Targeted literature search \& machine 
	learning on Aim 2 datasets \\
	\hline
	Miscellaneous functions (acid-tolerant, mucus-degrading, etc) & 
	Targeted literature search, unsupervised PiCRUST clustering, 
	genome mining \\
	\hline 
\end{tabular}
\caption{Possible approaches to define microbe sets of interest}\label{tab:microbe_set_categories}
%\end{wraptable}
\end{table}



This work will be the beginning of what will hopefully become a new 
approach to interpreting 16S datasets - moving the field from asking 
simply ``what's different?'' toward a more critical interpretation of 
``why are things different?''